
\newtcbtheorem[auto counter,number within=section]{theorem}%
{Theorem}{
    fonttitle=\upshape, 
    fontupper=\upshape,
    boxrule=0pt,
    leftrule=3pt,
    arc=0pt,auto outer arc,
    colback=white,
    colframe=orange,
    colbacktitle=white,
    coltitle=orange,
    oversize,
    enlarge top by=1mm,
    enlarge bottom by=1mm,
    enhanced jigsaw,
    interior hidden, 
    before skip=12pt,
    overlay={
      \draw[line width=1.5pt,orange] (frame.north west) -- (frame.south west);
    }, 
    frame hidden}{theorem}
    
\newtcbtheorem[]{exercise}%
{Exercise}{
      theorem name,
    fonttitle=\upshape, 
    fontupper=\upshape,
    boxrule=0pt,
    leftrule=3pt,
    arc=0pt,auto outer arc,
    colback=white,
    colframe=pink,
    colbacktitle=white,
    coltitle=pink,
    oversize,
    enlarge top by=1mm,
    enlarge bottom by=1mm,
    enhanced jigsaw,
    interior hidden, 
    before skip=12pt,
    after skip=0pt,
    overlay={
      \draw[line width=1.5pt,pink] (frame.north west) -- (frame.south west);
    }, 
    frame hidden}{exercise}

\newtcbtheorem[auto counter,number within=section]{lemma}%
{Lemma}{
    fonttitle=\upshape, 
    fontupper=\upshape,
    boxrule=1pt,
    toprule=0pt,
    leftrule=3pt,
    arc=0pt,auto outer arc,
    colback=white,
    colframe=orange,
    colbacktitle=white,
    coltitle=orange,
    oversize,
    enlarge top by=1mm,
    enlarge bottom by=1mm,
    enhanced jigsaw,
    interior hidden, 
    before skip=12pt,
    after skip=0pt,
    overlay={
      \draw[line width=1.5pt,orange] (frame.north west) -- (frame.south west);
    }, 
    frame hidden}{lemma}
    
\newtcbtheorem[auto counter,number within=section]{example}%
{Example}{
    fonttitle=\upshape, 
    fontupper=\upshape,
    boxrule=1pt,
    toprule=0pt,
    leftrule=3pt,
    arc=0pt,auto outer arc,
    colback=white,
    colframe=green,
    colbacktitle=white,
    coltitle=green,
    oversize,
    enlarge top by=1mm,
    enlarge bottom by=1mm,
    enhanced jigsaw,
    interior hidden, 
    before skip=12pt,
    overlay={
      \draw[line width=1.5pt,green] (frame.north west) -- (frame.south west);
    }, 
    frame hidden}{example}
    
\newtcbtheorem[]{important}%
{Wichtig}{
    fonttitle=\upshape, 
    fontupper=\upshape,
    boxrule=0pt,
    leftrule=3pt,
    arc=0pt,auto outer arc,
    colback=white,
    colframe=pink,
    colbacktitle=white,
    coltitle=pink,
    oversize,
    enlarge top by=1mm,
    enlarge bottom by=1mm,
    enhanced jigsaw,
    interior hidden, 
    before skip=12pt,
    overlay={
      \draw[line width=1.5pt,pink] (frame.north west) -- (frame.south west);
    }, 
    frame hidden}{important}
    
\renewcommand{\baselinestretch}{1.4} 
\makeatletter
\let\old@rule\@rule
\def\@rule[#1]#2#3{\textcolor{lightgrey}{\old@rule[#1]{#2}{#3}}}
\makeatother
